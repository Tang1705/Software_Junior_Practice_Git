\section{项目概述}

\subsection{项目背景}
    根据中国互联网络信息中心(CNNIC)第40次《中国互联网络发展状况统计报告》显示,截至2017年6月,中国网民规模达到7.51亿,互联网普及率为54.3%,超过全球平均水平4.6个百分点。伴随着高歌猛进地互联网化,以微博、微信为代表的网络社区成为了新的最重要的舆论场。微信在2016年底共计8.89亿月活用户,公众号平台超过1000万个,微博在2017年初也实现了月活用户3.4亿人次,面对错综复杂的舆论阵地,舆情产业是信息服务行业在大数据时代的又一轮升级产业,通过研究用户及其相关舆情,将有价值的信息传递给用户,最终帮助用户解决实际问题。
 \\经过十多年的发展,以政府部门为主导的舆情管理市场进入了高速成长期,年均增幅50%以上,达到了超百亿元的规模。舆情产业已成为一个多门类、复合型的知识密集产业,形成了以“官商媒教”为背景的产业格局,具备了“政用产学研”的产业链生态,其产品及服务涵盖了软件支持、知智体系、风险管理、危机应对等一系列内容,产业形态已初步具备。
 \\政府在舆情产业中起着主导作用。政府通过制定积极政策对舆情加以引导,同时以自身需要,作为大客户推动舆情领域发展,媒体尤其是官媒、新媒体,在舆情产业的发展中起着某种先锋作用。舆情管理在政策和市场双重因素的刺激下,基本形成了“官商媒教”的多元化产业格局,形成了政府、媒体、教育科研、商业软件四大背景的行业格局。
 \\对于影视作品来说,舆论与口碑在很大程度上决定了影视作品的成功与否。影视舆情信息在互联网上客观存在,形成了与影视剧内容产品相伴而生的衍生信息产品,形成影视剧不同传播时期的先期舆情(影视剧首轮发行期前)、同期舆情(首轮发行期同步)、后期舆情(首轮播映后)、长尾舆情。先期舆情和同期舆情在一定程度上作用于影视剧作品的观影期待,影响观看意向、初步评价、后期评价、口碑评分等,进而影响收视率和票房。因此,舆论导向的监测、分析与预警对于影视行业来说十分重要。

\subsection{业务概述}
\begin{table}
    \centering
    \begin{tabular}{|l|l|l|l|}
    \hline
        页面 & 序号 & 功能模块 & 功能概述 \\ \hline
        首页 & 1 & 登录/注册 & 用户进行登录/注册 \\ \hline
         & 2 & 热点舆情实时展示 & 用户可浏览当前热点舆情事件 \\ \hline
         & 3 & 功能入口 & 用户可查看系统功能 \\ \hline
        个人中心 & 4 & 账号信息 & 用户查看自己的账号信息 \\ \hline
         & 5 & 历史数据 & 用户查看自己的历史记录 \\ \hline
         & 6 & 近期关注 & 用户查看近期关注的热点 \\ \hline
         & 7 & 系统消息 & 用户查看收到的系统消息 \\ \hline
         & 8 & 历史分析报告 & 企业用户获得权限后可查看历史分析报告 \\ \hline
         & 9 & 预警详情 & 企业用户获得权限可查看预警详情 \\ \hline
        个人用户 & 10 & 筛选影视舆情事件 & 查询想要了解的舆情事件 \\ \hline
         & 11 & 添加关注 & 对感兴趣的热点事件添加关注 \\ \hline
        企业用户 & 12 & 查看分析数据 & 查看系统对某一事件的分析数据 \\ \hline
        舆情事件 & 13 & 事件详情 & 当前事件的关系网络,关注度等信息 \\ \hline
        后台 & 14 & 系统管理 & 管理员对系统进行维护 \\ \hline
    \end{tabular}
\end{table}

\subsection{技术概述}
\subsubsection{开发工具}
硬件:PC端、移动端

软件:浏览器 IE9 及以上等主流浏览器;Windows 操作系统;hadoop 框架,MongoDB 数据库
\subsubsection{技术路线}
浏览器端:前端采用 Vue.js 实现。

前后端交互:客户端通过Http请求服务器接口,服务器根据对应参数,返回JSON数据,客户端解析JSON数据,进行数据显示。

后台服务器:后端采用 Spring + Spring MVC +Hadoop 框架实现。