\section{系统非功能性需求分析}
\subsection{性能}
\textbf{吞吐量}
\begin{enumerate}
\item 每日最大成交数3000笔业务。
\item 平均交易并发数为20,最大交易并发数为50。
\item 估计用户数为1万人,每天登录用户数为3000左右,网络的带宽为100M带宽。
\item 系统可以同时满足10,000个用户请求,并为25,000个并发用户提供浏览功能。
\end{enumerate}
\textbf{响应时间}
\begin{enumerate}
\item 在95\%的情况下,一般时段响应时间不超过1.5秒,高峰时段不超过4秒。
\item 定位系统从点击到第一个界面显示出来所需要的时间不得超过300毫秒。
\item 在网络畅通时,拨号连接GPRS网络所需时间不得超过5秒。
\item 在网络畅通时,电子地图刷新时间不超过10秒。
\item 在推荐配置环境下:登录响应时间在2秒内,刷新栏目响应时间在2秒内,刷新条目分页列表响应时间2秒内,打开信息条目响应时间1秒内,刷新部门、人员列表响应时间2秒内。
\item 在非高峰时间根据编号和名称特定条件进行搜索,可以在3秒内得到搜索结果。
\end{enumerate}
\textbf{精度}
\begin{enumerate}
\item 定位精度误差不超过80米。
\item 当通过互联网接入系统的时候,期望在编号和名称搜索时最长查询时间$<$15秒。
\item 计算的精确性到小数点后7位。
\end{enumerate}
\textbf{资源使用率}
\begin{enumerate}
\item CPU占用率 $\leq$50\%。
\item 内存占用率 $\leq$50\%。
\end{enumerate}
\subsection{安全性}
\begin{enumerate}
\item 严格权限访问控制,用户在经过身份认证后,只能访问其权限范围内的数据,只能进行其权限范围内的操作。
\item 不同的用户具有不同的身份和权限,需要在用户身份真实可信的前提下,提供可信的授权管理服务,保护数据不被非法/越权访问和篡改,要确保数据的机密性和完整性。
\item 提供运行日志管理及安全审计功能,可追踪系统的历史使用情况。
\item 能经受来自互联网的一般性恶意攻击。如病毒(包括木马)攻击、口令猜测攻击、黑客入侵等。
\item 至少99\%的攻击需要在10秒内检测到。
\end{enumerate}
\subsection{可靠性}
影视舆情分析、预警与监测系统要求能在 24 小时内一直稳定运行。当发生某些热点事件时,随着系统使用人数的不断增长,客户端向服务端发送的请求也越来越多,如果只有一台应用服务器来接受来自很多客户端的请求,一旦资源耗尽,发生宕机的可能性很大,所以系统的可靠性十分重要。

\textbf{易恢复性}
本系统发生故障后,系统应重建其性能水平并恢复直接受影响数据的能力。发布新版本时,要做好回滚方案,以备异常紧急处理。同时做好备份,系统监控的字段以及历史查询信息误删除时可进行恢复。

\textbf{容错性}
在系统出错时,不影响用户的行为操作与数据。在设计数据库的时候,可以进行冗余设计,采用主从数据库的方式,把读操作和写操作分离,部署在不同的服务器上,从数据库主要用来查询数据用,不进行写入操作,主数据库为写库,用来写入和更新数据,每次当主数据库有写的操作时,数据同步到从数据库去。设计多个从数据库,即拥有了多个容灾副版本,当主数据库服务器宕机的时候,马上切换到其中一台正常运行的从数据库服务器去,提高了整个系统的容错性。

\textbf{成熟性}
系统故障率需要保持在一定水平以下。在设计系统的时候,可以考虑部署一台负载均衡服务器,把请求合理的分配到多台应用服务器上去,达到资源的合理分配。Nginx 是一个可以用来做反向代理的负载均衡的轻量级软件,占用内存小,带宽低,并发连接数大,配置简单,功能强大,还可以缓存静态资源,例如图片等。具体设计时,可以部署一台 Nginx反向代理服务器作为接受客户端请求的统一访问接口点,根据业务量的多少来设置集群服务器的个数,起到负载均衡的效果。

\subsection{易用性}
易用性是以用户为中心,结合视觉、交互、情感等综合感受,使产品符合用户习惯的能力以及其对使用的期望。 它会对用户使用产品的生产效率、错误率以及用户对新产品的接收程度产生很大的影响。

\textbf{易学习性}
系统学习成本低。本系统无需学习即可使用。

\textbf{易操作性}
系统建设过程中,系统涵盖了完整的业务需求。模块在组件各部分之间实现信息的顺畅流动,系统具有连贯性,交互设置合理,功能明确清晰。

\textbf{用户错误防御机制}
我们设计了上百种错误防御机制,系统遇到错误的输入时会触发检测机制,提示用户输入错误,防止造成系统崩溃影响用户使用体验。

\textbf{用户界面美观}
本系统采用大屏数据展示界面,进行了web界面原型设计并经用户反复确认,并且通过了性能测试、系统测试及用户接受测试。

\subsection{可维护性与可扩展性}
在设计与开发本系统时,采用了以下的方法:

\begin{enumerate}
\item 程序级别的可扩展性,主要通过参数化配置程序低级别的可扩展性。
\item 高度可配置性,包括各种属性文件和 XML 配置文件。
\item 脚本。脚本是扩展复杂功能的利器,但对客户的要求也比较高。通常应该是面向开发人员的工具产品或在产品部署之前由现场实施人员来完成。
\item 插件系统或模块化平台。插件系统平台从理论上提供了无数的可扩展性。在获知辽宁省标准化研究院的标准管理业务模型的基础上,再考虑上述的一般软件的扩展性需求,就是本系统的可扩展性需求。
\end{enumerate}
	
使系统具有可维护性与可扩展性,主要表现在:

\textbf{模块性}
当某些业务流程变动多,此时将系统功能模块化,支持灵活配置,有利于减少重复开发量。本系统高度内聚各个功能模块,通过抽象类设计、接口设计等方式减少模块的耦合。引用外部方法时,我们将外部方法作为变量传入本函数作用域,再加以使用。降低了模块代码间的耦合性,有利于软件的后期维护。同时,本系统在设计每个功能模块时,如查询、监测、情感分析等等,尽可能做到了划分明确,降低彼此间的依赖性。

\textbf{可复用性}
查询人物、查询影视作品和查询事件的模块类似,系统开发时进行了统一设计,在需要用到的地方可进行微调然后调用,开发过程中注重提高模块的可重用性。

\textbf{易分析性}
本系统每隔一段时间会生成系统运行日志,可追踪系统的历史使用情况,易诊断缺陷或失败原因。


