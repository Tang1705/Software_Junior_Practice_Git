\section{四、系统非功能性需求分析}
\subsection{4.1 性能}
\paragraph{吞吐量}
1.每日最大成交数3000笔业务。
2.平均交易并发数为20,最大交易并发数为50。
3.估计用户数为1万人,每天登录用户数为3000左右,网络的带宽为100M带宽。
4.系统可以同时满足10,000个用户请求,并为25,000个并发用户提供浏览功能。
\paragraph{响应时间}
1.在95%的情况下,一般时段响应时间不超过1.5秒,高峰时段不超过4秒。
2.定位系统从点击到第一个界面显示出来所需要的时间不得超过300毫秒。
3.在网络畅通时,拨号连接GPRS网络所需时间不得超过5秒。
4.在网络畅通时,电子地图刷新时间不超过10秒。
5.在推荐配置环境下:登录响应时间在2秒内,刷新栏目响应时间在2秒内,刷新条目分页列表响应时间2秒内,打开信息条目响应时间1秒内,刷新部门、人员列表响应时间2秒内。
6.在非高峰时间根据编号和名称特定条件进行搜索,可以在3秒内得到搜索结果。
\paragraph{精度}
1.定位精度误差不超过80米。
2.当通过互联网接入系统的时候,期望在编号和名称搜索时最长查询时间<15秒。
3.计算的精确性到小数点后7位。
\paragraph{资源使用率}
1.CPU占用率<=50\%。
2.内存占用率<=50\%。

\subsection{4.2 安全性}
1.严格权限访问控制,用户在经过身份认证后,只能访问其权限范围内的数据,只能进行其权限范围内的操作。
2.不同的用户具有不同的身份和权限,需要在用户身份真实可信的前提下,提供可信的授权管理服务,保护数据不被非法/越权访问和篡改,要确保数据的机密性和完整性。
3.提供运行日志管理及安全审计功能,可追踪系统的历史使用情况。
4.能经受来自互联网的一般性恶意攻击。如病毒(包括木马)攻击、口令猜测攻击、黑客入侵等。
5.至少99\%的攻击需要在10秒内检测到。

\subsection{4.3 可靠性}
影视舆情分析、预警与监测系统要求能在 24 小时内一直稳定运行。当发生某些热点事件时,随着系统使用人数的不断增长,客户端向服务端发送的请求也越来越多,如果只有一台应用服务器来接受来自很多客户端的请求,一旦资源耗尽,发生宕机的可能性很大,所以系统的可靠性十分重要。
\paragraph{易恢复性}
本系统发生故障后,系统应重建其性能水平并恢复直接受影响数据的能力。发布新版本时,要做好回滚方案,以备异常紧急处理。同时做好备份,系统监控的字段以及历史查询信息误删除时可进行恢复。
\paragraph{容错性}
在系统出错时,不影响用户的行为操作与数据。在设计数据库的时候,可以进行冗余设计,采用主从数据库的方式,把读操作和写操作分离,部署在不同的服务器上,从数据库主要用来查询数据用,不进行写入操作,主数据库为写库,用来写入和更新数据,每次当主数据库有写的操作时,数据同步到从数据库去。设计多个从数据库,即拥有了多个容灾副版本,当主数据库服务器宕机的时候,马上切换到其中一台正常运行的从数据库服务器去,提高了整个系统的容错性。
\paragraph{成熟性}
系统故障率需要保持在一定水平以下。在设计系统的时候,可以考虑部署一台负载均衡服务器,把请求合理的分配到多台应用服务器上去,达到资源的合理分配。Nginx 是一个可以用来做反向代理的负载均衡的轻量级软件,占用内存小,带宽低,并发连接数大,配置简单,功能强大,还可以缓存静态资源,例如图片等。具体设计时,可以部署一台 Nginx反向代理服务器作为接受客户端请求的统一访问接口点,根据业务量的多少来设置集群服务器的个数,起到负载均衡的效果。

\subsection{4.4 易用性}
易用性是以用户为中心,结合视觉、交互、情感等综合感受,使产品符合用户习惯的能力以及其对使用的期望。 它会对用户使用产品的生产效率、错误率以及用户对新产品的接收程度产生很大的影响。
\paragraph{易学习性}
系统学习成本低。本系统无需学习即可使用。
\paragraph{易操作性}
系统建设过程中,系统涵盖了完整的业务需求。模块在组件各部分之间实现信息的顺畅流动,系统具有连贯性,交互设置合理,功能明确清晰。
\paragraph{用户错误防御机制}
我们设计了上百种错误防御机制,系统遇到错误的输入时会触发检测机制,提示用户输入错误,防止造成系统崩溃影响用户使用体验。
\paragraph{用户界面美观}
本系统采用大屏数据展示界面,进行了web界面原型设计并经用户反复确认,并且通过了性能测试、系统测试及用户接受测试。


