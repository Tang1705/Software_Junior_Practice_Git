\newpage
\section{数据库设计}

\subsection{大数据分析模块}
大数据分析模块负责进行数据的处理,它的整体框架图是这样的:
\begin{figure}[!htbp]
	\centering
	\includegraphics[scale=1.2]{image/d1.png}
	\caption{大数据分析模块}
\end{figure}

\subsection{数据存储系统的设计与实现}
\textbf{介绍}
数据存储系统是大数据可视化的基础,是整个系统的根脉。
对于存储的大量数据进行统计,计算和处理通常需要以小时和天为记的时间,而本系统的用户会根据需要对于已经处理好的数据进行可视化定制。所以数据存储系统的设计需要满足以下用户需求:
\begin{itemize}
	\item 对已处理的数据进行可视化渲染
	\item 对结构简单的数据进行多样化的渲染,比如同时渲染成柱状图和饼图
	\item 对渲染的低延时需求,需要“即做即画”
	\item 对某组数据进行快速索引,通过少量关键字迅速检索需要的数据
\end{itemize}

基于以上需求特点,本系统采用Hadoop生态中的HBase加Mysql数据库存储方案,该方案的基本流程是:将所有需要的数据,放入hdfs中,然后用hadoop进行分析,得出来最终的结果数据,再放入mysql数据库表中。前端接口再查询访问这些最终的结果数据。
具有以下特点:
\begin{itemize}
	\item 擅长存储半结构化数据,存储结构灵活
	\item 面向列的设计使得一张表中可以存储数以万计的不同图表数据
	\item 可以提供低延时查询,通过行键的查询延时在1ms内
\end{itemize}

该方案的好处:
\begin{itemize}
	\item 所有的数据放入了hdfs中,而不放入线上mysql数据库中,这样在统计分析数据时,就不会影响正在操作数据库的用户。
	\item 利用了hadoop的“分布”式“计算”框架的优势,可以像多核cpu一起共同计算一样,这样分析速度会快。
	\item 减少了线上mysql数据库的负担。
	\item 所有需要的数据,都放在hdfs中,这样让相关人员的思路也会觉得清晰。
\end{itemize}

\newpage
\textbf{HBase的基本存储方式}
\begin{itemize}
\item 表名
\end{itemize}

\begin{table}[!htbp]
	\centering
	\caption{表名}
	\label{tab:my-table}
	\begin{tabular}{|p{3cm}|p{10cm}|}
		\hline
		\rowcolor[HTML]{DAE8FC} 
		行键 & 列族 \\ \hline
		通过键值检索行 & 在创建表时定义所有列族  \\ \hline
		通常存储全表最关键的索引信息 & 每行中每个列族可以存储任意多列 \\ \hline
		一个行键的大小最大为单元格大小 & 一行和一个列族和一个列对应一个单元格 \\ \hline
        行键在全表中唯一 & 可以重复,不同的单元格用时间戳区分 \\ \hline
        每一行的每个列族下的列都不一定一样 & 可以重复,不同的单元格用时间戳区分,查询时一般查询最新单元格 \\ \hline
	\end{tabular}
\end{table}



