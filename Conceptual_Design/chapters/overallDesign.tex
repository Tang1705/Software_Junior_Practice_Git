\section{总体设计}
\subsection{概述}
本系统为基于大数据的舆情分析与预警系统,当今社会,互联网蓬勃发展,我们正处于一个一切皆有可能的大变革时代,纸媒、 微博、微信、APP正
在随时随地地影响着人们的生活,舆情场也随之改变,社会化媒体尤其是微博成为舆情爆发的主要阵地。本系统通过收集这些社会化媒体的数据,
对当前比较热门的话题等进行舆情分析,并对该舆情的发展方向进行预测,对可能出现的负面影响预警。\newline
本系统设计为 B / S 架构,前端采用 HTML 5 + CSS 3 + Angular Js 实现,后端采用目前在 Web 设计中普遍使用的 SSM(Spring + Spring MVC +MyBatis)框架实现,
本设计使系统具有优秀的解耦性,并大大增强了系统的可扩展性和可维护性。
\subsection{系统环境描述}
\subsubsection{运行环境}
\subsubsubsection{软件环境}
\begin{table}
\begin{tabular}{llll}
分类&名称&版本&语言
操作系统&Windows&Windows7/8/XP/10&简体中文\\
数据库平台&hadoop/spark&Windows7/8/XP/10&简体中文\\
应用平台&Tomcat&8.0&简体中文\\
浏览器端&IE&IE9及以上&简体中文\\
\end{tabular}
\end{table}
\subsubsubsection{硬件环境}
\begin{table}
\begin{tabular}{lll}
应用及服务器&最低配置&推荐配置\\
Mem&8G&32G\\
HD&160G&600G\\
\end{tabular}
\end{table}
\subsubsection{开发环境}
\subsubsubsection{开发机器软件环境}
\begin{table}
\begin{tabular}{llll}
分类&名称&版本&语言
操作系统&Windows&Windows7/8/XP/10&简体中文\\
数据库平台&hadoop/spark&Windows7/8/XP/10&简体中文\\
应用平台&Tomcat&8.0&简体中文\\
开发平台&JDK&1.8&English\\
\end{tabular}
\end{table}
\subsubsubsection{开发机器硬件环境}
\begin{table}
\begin{tabular}{lll}
开发机器&最低配置&推荐配置\\
Mem&4G&12G\\
HD&160G&600G\\
\end{tabular}
\end{table}
\subsection{系统总体结构设计}

\subsubsection{系统业务层次图}


\subsubsection{模块功能介绍}


\subsubsection{模块间接口设计}
